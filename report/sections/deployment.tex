\section{Deployment}
Il pregetto è condiviso su un repository GitHub al seguente link:
\newline{\\}
\noindent
\underline{\href{https://github.com/brteo/TradingVG/}{https://github.com/brteo/TradingVG/}}
\newline{\\}
\noindent
Clonare il progetto sul proprio dispositivo e seguire la indicazione.
\newline{\\}
\noindent
\textbf{Requisiti}
\begin{itemize}
	\item{\underline{\href{https://docs.npmjs.com/downloading-and-installing-node-js-and-npm}{node and npm}}}
	\item{\underline{\href{https://docs.docker.com/compose/install/}{docker compose}}}
\end{itemize}
\noindent
\textbf{Installazione}
\begin{enumerate}
	\item{installare i pacchetti npm all'interno della cartella "api"}
\begin{lstlisting}[language=bash]
	npm i
\end{lstlisting}		
	\item{installare i pacchetti npm all'interno della cartella "web"}
\begin{lstlisting}[language=bash]
	npm i --legacy-peer-deps
	# add flag --legacy-peer-deps with npm version > 7.0
\end{lstlisting}
	\item{avviare docker dalla cartella principale}
\begin{lstlisting}[language=bash]
	docker compose up -d
\end{lstlisting}
	\item{una volta che docker è avviato, inizializzare i seed del database all'interno della cartella "api"}
\begin{lstlisting}[language=bash]
	npm run seed
\end{lstlisting}
\end{enumerate}

\bigbreak
\noindent
\textit{Per qualsiasi altra informazione o approfondimento si può consultare il README presente sul repo.}