\section{Requisiti}
Il progetto si pone l'obiettivo di realizzare un portale web accessibile a chiunque, 
all'interno del quale utenti registrati possono creare e scambiarsi NFT.

Le opere caricate in formato digitale dovranno essere le protagoniste del sito web
quindi si dovrà realizzare un'estetica dal design minimalista ma allo stesso tempo funzionale.
Considerando che almeno il 55\% degli utenti naviga sul web con dispositivi mobile \cite{require1}, 
si dovrà inoltre prestare attenzione nello realizzare un'interfaccia responsive che si adatti a tutte le situazioni. 

Specifiche funzionali:
\begin{itemize}
	\item Registrazione utenti con possibilità di definire l'account su blockchain
	\item Accesso utenti con email e password
	\item Gestione di diversi ruoli utente: \textit{amministratore} ed \textit{utente base}
	\item Possibilità di creare nuovi NFT con \textit{titolo}, \textit{descrizione}, \textit{immagine}, \textit{categoria} e \textit{tags} salvando le informazioni necessarie su blockchain.
	\item Possibilità di creare una nuova asta associata all'NFT con \textit{prezzo di partenza}, \textit{data di scadenza} e \textit{descrizione}.
	\item Home page dove vengono visualizzati di tutti gli NFT, in vendita e non, con possibilità di ricerca
	\item Pagina dettaglio NFT e nel caso ci fosse un'asta attiva, possibilità di partecipare e visualizzare lo storico delle puntate 
	\item Profilo utente con lista di NFT creati e posseduti
	\item Multilingua
\end{itemize}

Per quanto riguarda la scelta della tecnlogia di blockchain da utilizzare 
si dovrà fare una fase di studio per trovare quella più adatta alle esigenze 
tra quelle che supportano gli NFT e smart contract. 
