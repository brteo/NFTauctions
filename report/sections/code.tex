\section{Codice}
Durante la fase iniziale di sviluppo ci siamo preoccupati di realizzare delle funzionalità lato BackEnd e FrontEnd 
che poi ci permettessero di standardizzare il metodo di lavoro e velocizzare la scrittura del codice,
concentrandoci così sulla logica dell'applicativo.
\bigbreak
\noindent
Di seguito sono riportati alcuni gli aspetti che riteniamo più rilevanti lato codice.

\subsection{BackEnd}

\textbf{Uniformate le risposte di express}
\bigbreak
\noindent
Per prima cosa abbiamo cercato di uniformare tutte le risposte di Express per fare in modo di gestire agevolmente gli error, 
creado un middleware che si occupasse di rispondere effettivamente alle chiamate.

In questo modo, all'interno dei vari controllers che gestiscono le chiamate ci siamo permessi di utilizzare una funzione 
che generasse la corretta risposta impostando i giusti parametri tra cui lo status code.

Alcuni esempi: 

\begin{lstlisting}[language=java]
	// invio errore generale con status code 500
	next(ServerError()); 

	// invio errore risorsa non trovata con status code 404
	next(NotFound());

	// invio dati richiesti in formato Json con status code 200
	next(SendData({...}));
\end{lstlisting}

\bigbreak
\noindent
\textbf{Autenticazione}
\bigbreak
\noindent
Abbiamo realizzato da zero il processo di autenticazione con l'ausilio dei pacchetti \textbf{passport} e \textbf{jwt}.

Durante la fase di login, vengono inviate le credenziali email e password.
Se le credenziali sono corrette vengono generati due JSON Web Token:
\begin{itemize}
	\item \textbf{authToken}: JWT contenente nel payload l'identificativo utente su database (durata 5 minuti)
	\item \textbf{refreshToken}: JWT contenente nel payload il refresh token generato tramite uuid e salvato sul database in relazione all'utente (durata 30 giorni)
\end{itemize}
Per cercare la massima sicurezza ed evitare che il client salvi questi token riservati all'interno del browser, in posti accessibili, 
abbiamo deciso di utilizzare i cookie http-only: cookie settati dal server e che il browser storicizza in uno spazio di memoria non accessibile dall'esterno.

Una volta che il client è in possesso del authToker, questo viene inviato per ogni richiesta al server che, 
tramite un middleware creato ad hoc, verifica la correttezza nelle route degli entpoint che richiedono l'accesso per essere richiamati e,
in caso di errore, viene automaticamente restituito l'errore 401 (\textit{Unauthorized}) evitando di coinvolgere il controller ed interrompendo la richiesta.

\begin{lstlisting}[language=java]
	router.get('/check', isAuth, check);
\end{lstlisting}

Quando l'authToken scade, il client può richiedere automaticamente un nuovo token di autenticazione utilizzando il refreshToken.
Ogni utente può avere più di un refreshToken attivo in quanto è previsto che possa fare accesso da più dispositivi e quindi sono salvati su database sotto forma di array.

Infine è stato aggiunto un ulteriore controllo di sicurezza: nel caso venga inviato un refreshToken non più valido, 
l'utente viene bloccato in quanto è stato probabilmente soggetto di furto del token. 
Infatti l'unico modo per non avere un refreshToken valido è che qualcuno lo abbia utilizzato in precedenza e che quindi è stato rimosso sul database.
Di conseguenza, se avviene una richiesta di questo tipo, significa che un malintenzionato ha rubato il token 
e provato ad utilizzare per ottenere accesso dopo che è stato già utilizzato e quindi invalitato dall'utente o viceversa, 
in entrambi i casi l'utente viene bloccato.

\bigbreak
\noindent
\textbf{Role-Based Access Control}
\bigbreak
\noindent
Sono stati realizzati ad hoc i Role-Based Access Control (RBAC) per gestire i permessi delle azioni CRUD della RestApi.
Abbiamo previsto che esistano due tipi di utente: \textbf{admin} e \textbf{user}.

Ogni ruolo ha un file Json che descrive per ogni risorsa la relativa policy, 
definendo per ogni azione di tipo CRUD se ha il permesso 
di manibilare o richiedere qualsiasi informazioni ("\textit{any}") o soltanto le proprie ("\textit{own}").

\begin{lstlisting}[language=java]
	// esempio ruolo "admin" per le risorse "users"
	{
		"users": { 
			"create:any": ["*"],
			"read:any": ["*"],
			"update:any": ["*"],
			"delete:any": ["*"]
		}
	}

	// esempio ruolo "user" per le risorse "users"
	{
		"users": {
			"read:own": ["*"],
			"update:own": ["*","!active"] 
			// can't change active field
		}
	}
\end{lstlisting}

A questo punto ci è bastato creare un middleware di controllo da inserire nelle richieste che necessitano di verifica del ruolo per essere eseguite.
Questo middleware è sempre aggiunto in successione a quello dell'autenticazione 
in quanto la verifica del ruolo necessita sicuramente di un'autenticazione che valida l'utente che effettua la richiesta e,
in caso di errore, viene automaticamente restituito l'errore 403 (\textit{Forbidden}) evitando di coinvolgere il controller ed interrompendo la richiesta.

\begin{lstlisting}[language=java]
	router.route('/')
		.get(isAuth, rbac('users', 'read:any'), controller.get);
\end{lstlisting}

\bigbreak
\noindent
\textbf{Plugins Mongoose}
\bigbreak
\noindent
Per rendere il lavoro sui controller più automatizzato possibile,
abbiamo realizzato due plugin per Mongoose: \textbf{SoftDelete} e \textbf{dbFileds}.

Il plugin \textbf{SoftDelete} può essere aggiunto ad un Model del database per definire che la cancellazione di un documento di quella collezione
è di tipo logica e non reale.
Durante la progettazione del database, abbbiamo deciso di applicare questa tecnica, su alcune collezioni del database,
che ci permette di cancellare le risorse definendo i campi aggiuntivi \textit{deleted} e \textit{deletedAt}. 
In questo modo possiamo mantenere lo storico su database e ripristinare eventuali errori umani di cancellazione dati
in quanto la risorsa non viene effettivamente cancellata.

\begin{lstlisting}[language=java]
	const softDelete = require('../helpers/softDelete');
	const { Schema } = mongoose;

	const schema = Schema({
		// schema
	});
	schema.plugin(softDelete);
\end{lstlisting}

Per evitare di ricordare di aggiungere in ogni query sul database il filtro "\textit{deleted: false}",
è stato automaticamente inserito in tutte le query del modello attraverso il plugin
a meno che non sia esplicitamente definito dallo sviluppatore.

Per quanto riguarda il plugin \textbf{dbFileds}, ci siamo accorti durante lo sviluppo di volere un automatismo per definire quali campi sono pubblici 
e che quindi possono essere inviati dalle risposte della RestApi. 
Ad esempio il campo \textit{password} sulla collezione degli utenti non deve mai ritornare come risposta.

Di conseguenza abbiamo creato un plugin che ci permettese di definire una o più liste di campi pubblicida poter selezionare durante le query:

\begin{lstlisting}[language=java]
	const softDelete = require('../helpers/softDelete');
	const { Schema } = mongoose;

	const schema = Schema({
		// schema
	});

	schema.plugin(dbFields, {
		public: ['_id', 'nickname', 'pic', 'role', 'lang'],
		profile: ['_id', 'nickname', 'pic', 'header', 'bio'],
		cp: ['_id', 'email', 'account', 'nickname', 'name', 'lastname', 'pic', 'lang', 'role', 'active']
	});
\end{lstlisting}

\bigbreak
\noindent
\textbf{Aggiunta di una nuova puntanta sull'asta}
\bigbreak
\noindent
Particolare attenzione è stata riposta all'aggiunta di una nuova puntata su un'asta attiva.

Nella progettazione del database, avevamo deciso di lasciare delle informazioni ridondanti all'interno dell'asta (collezione \textit{auction}),
salvando le ultime 10 puntate andate a buon fine. 
Ovviamente ognuna di queste puntate viene salvata anche nella relativa collezione \textit{bets} con il riferimento all'asta.

Il problema principale era però l'eventuale concorrenza data da più richieste contemporantemente
che si sarebbe presentato qualora avessimo eseguito due query in successione:
\begin{itemize}
	\item una per la verifica del prezzo della puntata, che deve essere maggiore di quello attuale
	\item una per l'effettivo aggiornamento della risorsa
\end{itemize}

Anche se JavaScript è un linguaggio asincrono single thread, non abbiamo la garanzia che ogni callback delle query abbia la giusta sequenza.
Potrebbe capitare che viene lanciato due volte il controllo sul prezzo, da due client diversi, 
prima di inserire effettivamente la puntata e quindi permettere a due utenti diversidi effettuare anche la stessa puntata.

Per risolvere tale problema, abbiamo approfondito la documentazione di Mongoose 
dove abbbiamo scoperto che l'operazione \textbf{findOneAndUpdate} ci viene garantita come atomica.

\begin{lstlisting}[language=java]
	const auction = await Auction.findOneAndUpdate(
		{ _id: auctionId, price: { $lt: betPrice } },
		{
			price: betPrice,
			$push: {
				lastBets: {
					$each: [newBets],
					$position: 0,
					$slice: 10
				}
			}
		},
		{ new: true }
	).exec();
\end{lstlisting}

\clearpage

\subsection{FrontEnd}

\bigbreak
\noindent
\textbf{Context}
\bigbreak
\noindent
Tipicamente in React, i dati vengono passati dall'alto verso il basso (da genitore a figlio) tramite props, 
ma per alcune informazioni avevamo la necessità di renderle disponibili a tutti i componenti.

Per fare questo React ci permette di creare un \textit{contesto} ed utilizzarlo tramite l'hook \textbf{useContext}.
Quando un elemento del contesto viene modificato, tutti i componenti al suo interno vengono verificati ed eventualmente aggiornati, 
come avviene normalmente per lo stato o le props.

Abbiamo quindi realizzato un contesto che racchiude l'intera struttura del sito web dove condividere a tutti i componenti le seguenti informazioni/funzioni:
\begin{itemize}
	\item \textbf{logged}: informazioni sull'utente che ha effettuato l'accesso al sito, se non presente significa che l'utente è un ospite
	\item \textbf{setLogged} funzione per settare l'utente collegato una volta che ha effettuato l'accesso
	\item \textbf{isMobile}: identifica se il dispositivo di visualizzazione è di tipo smartphone oppure no
	\item \textbf{handleLogout}: funzione effettuare il logout dell'utente
	\item \textbf{showLogin}: identifica se la popup di accesso/registrazione è visibile
	\item \textbf{setShowLogin}: funzione per l'apertura della popup di accesso/registrazione
\end{itemize}

Nel momento in cui il contesto viene inizializzato ed visualizzato sulla pagina, 
tramite l'hook \textbf{useEffect} viene effettuato l'autologin: 
se l'utente aveva già navigato sul sito ed aveva effettuato l'accesso senza eseguire il logout,
allora probabilmente avrà un refreshToken (durata di 30 giorni) ancora impostato sui cookie http-only,
di conseguenza viene automaticamente provato a richiedere un'autenticazione e,
nel caso l'operazione vada a buon fine, 
l'utente si ritroverà loggato senza dover effettuare nuovamente l'accesso.
\bigbreak
Inoltre, all'interno del contesto, abbiamo sfruttato gli \textit{interceptors} di axios per intercettare automaticamente tutte le chiamate http di tutto il FrontEnd.
Grazie a questa funzionalità abbiamo gestito la seguente casistica: 
se l'utente è loggato e riceve l'errore 401 che può trovarsi nella situazione che l'authToken è scaduto (la sua durata è di 5 minuti).
In questo caso, viene automaticamente provata a richiedere una nuova autorizzazione tramite il refreshToken.
Se questa va a buon fine allora viene nuovamente effettuata la chiamata originaria evitandoci di prevedere questa casistica in ogni chiamata http fatta dal FrontEnd.

\bigbreak
\noindent
\textbf{Custom hook}
\bigbreak
\noindent
Avendo la necessità di salvare alcune informazioni utili al sistema sul localstorage del browser, 
abbiamo deciso di creare un Custom Hook che ci permettese di salvare le informazioni ed utilizzarle come lo stato all'interno dei componenti che lo necessitano.

\begin{lstlisting}[language=java]
	const useLocalStorage = (key, initValue) => {
		const [value, setValue] = useState(() => {
			try {
				const item = window.localStorage.getItem(key);
				return item ? JSON.parse(item) : initValue;
			} catch (err) {
				return initValue;
			}
		});

		const setStoredValue = newValue => {
			try {
				const valueToStore = newValue instanceof Function ? newValue(value) : newValue;
				setValue(valueToStore);
				window.localStorage.setItem(key, JSON.stringify(valueToStore));
			} catch (error) { }
		};

		return [value, setStoredValue];
	};
\end{lstlisting}
