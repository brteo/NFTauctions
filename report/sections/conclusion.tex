\section{Conclusioni}

Il progetto svolto è stato molto formativo. Ci ha permesso
di lavorare ad un progetto non banale per la realizzazione di un prototipo che 
con ancora qualche accorgimento e sviluppo potrebbe essere 
lanciato realmente come piattaforma sul web.

Durante lo svolgimento sono state messe alla prova non solo le capacità di programmazione ma anche di organizzazione in team
utilizzando un metodo di sviluppo Agile, cercando di seguire per quanto possibile un approccio "SCRUM".

Dal punto di vista tecnologico abbiamo cercato di approfondire molti aspetti non affrontati a lezione:
\begin{itemize}
	\item Abbiamo utilizzato React con un approccio puramente funzionale e non a classi come consigliato nella loro documentazione. 
	Compreso incrementalmente, mano a mano che si organizzavano e creavano componenti, come gestire correttamente il loro stato e come questo viene aggiornato da React.	
	\item Imparato ad utilizzare un database non relazionale come MongoDB, dove abbiamo dovuto imparare quali sono le "best practice"
		per la modellazione dei dati e come aggregarli tramite query facendo sempre attenzione alle performance.
	\item Strutturanto e testato una RestApi che segua i principi RESTful e che sia solida per eventuali sviluppi futuri.
	\item Studiato ed utilizzato la tecnologia blockchain, capendone i vantaggi e le limitazioni confrontando le principali tipologie che dominano il mercato.
\end{itemize}
 
Inoltre, sono state di rilevante importantanza le conoscenze provenienti da diversi corsi di laurea:
\begin{itemize}
	\item Sicurezza delle Reti: prestando attenzione ad aspetti di sicurezza e adottando le dovute tecniche nella
prevenzione di attacchi, soprattutto nella strutturazione dell'autenticazione utente, sia lato BackEnd che FrontEnd.
	\item Paradigmi di Programmazione e Sviluppo: applicando un approccio Test Driven per lo sviluppo della RestApi al fine di minimizzare eventuale debito tecnico.
  \item Programmazione Concorrente e Distribuita: sfruttando le conoscenze di programmazione asincrona basata su Event-Loop, per un corretto utilizzo di Promise ed Async/Await.
  \item Sistemi Distribuiti: sfruttando le conoscenze di base acquisite per poter utilizzare una Blockchain 
\end{itemize}

In conclusione, siamo pienamente soddisfatti del risultato ottenuto e delle
conoscenze e capacità acquisite nel cimentarci in questo progetto.