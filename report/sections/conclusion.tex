\section{Conclusioni}

Il progetto svolto è stato molto formativo. Ci ha permesso
di lavorare a un progetto non banale che potrebbe essere 
utilizzato come piattaforma reale su internet. Per questo sono 
state messe alla prova non solo le capacità di programmazione ma anche di 
organizzazione di un team. Sono state maturate competenze di lavoro di squadra utilizzando 
strumenti di supporto come Git Workflow.
Dal punto di vista tecnologico abbiamo cercato di approfondire molti aspetti
non affrontati a lezione: abbiamo utilizzato l’ultima versione del
framework React, imparato ad utilizzare il database non relazione MongoDB attraverso moongose, imparato a strutturare dei test per la RestAPI 
e soprattutto aver utilizzato una Blockchain che è una tecnologia moderna e per niente non banale.
Inoltre, sono state di rilevante importantanza le conoscenze provenienti da diversi corsi di laurea:
\begin{itemize}
	\item  Sicurezza delle Reti: prestando attenzione ad aspetti di sicurezza e adottando le dovute tecniche nella
prevenzione di attacchi come la riproduzione delle comunicazioni e nella memorizzazione delle password.
	\item Paradigmi di Programmazione e Sviluppo: utilizzando uno strumento come Git Workflow come per organizzare lo uno sviluppo Agile del team.
    \item Programmazione Concorrente e Distribuita: sfruttando le conoscen-
ze del modello di concorrenza basato su async/await, fondamentale nel backend.
    \item Sistemi Distribuiti: sfruttando le conoscenze di base acquisite per poter utilizzare una Blockchain 
\end{itemize}

In conclusione, siamo pienamente soddisfatti del risultato ottenuto e delle
conoscenze e capacità acquisite nel cimentarci in questo progetto.